\documentclass[a4paper]{jpconf}
\usepackage{graphicx}
\begin{document}
\title{Deployment of IPv6 only CPU resources at WLCG sites}

\author{A Dewhurst, A Sciaba, D Kelsey, R Nandakumar, C Grigoras, B Hoeft, F Prelz, E Martelli, D Traynor, K Hafeez, K Ohrenberg, R Voice, M Batik, T Finnern, S Fayer, T Froy, F Munoz, J Chudoba, D Rand, T Idiculla and U Tigerstedt}


\address{$^1$ STFC - Rutherford Appleton Lab. UK}

\ead{alastair.dewhurst@cern.ch}

\begin{abstract}
The fraction of internet traffic carried over IPv6 continues to grow rapidly. IPv6 support from network hardware vendors and carriers is pervasive and becoming mature. A network infrastructure upgrade often offers sites an excellent window of opportunity to configure and enable IPv6.

There is a significant overhead when setting up and maintaining dual stack machines, so where possible sites would like to upgrade their services directly to IPv6 only. In doing so, they are also expediting the transition process towards its desired completion. While the LHC experiments accept there is a need to move to IPv6, it is currently not directly affecting their work. Sites are unwilling to upgrade if they will be unable to run LHC experiment workflows. This has resulted in a very slow uptake of IPv6 from WLCG sites.

For several years the HEPiX IPv6 Working Group has been testing a range of WLCG services to ensure they are IPv6 compliant. Several sites are now running many of their services as dual stack. The working group, driven by the requirements of the LHC VOs to be able to use IPv6-only opportunistic resources, continues to encourage wider deployment of dual-stack services to make the use of such IPv6-only clients viable.

This paper will present the HEPiX plan and progress so far to allow sites to deploy IPv6 only CPU resources. This will include making experiment central services dual stack as well as a number of storage services. The monitoring, accounting and information services that are used by jobs also needs to be upgraded. Finally the VO testing that has taken place on hosts connected via IPv6 only will be reported.
\end{abstract}

\section{Introduction}

\section{IPv6 Peering and PerfSonar work}

\section{IPv6 only CPU resources}


\subsection{Acknowledgments}



\section*{References}
\begin{thebibliography}{9}
\bibitem{iopartnum} IOP Publishing is to grateful Mark A Caprio, Center for Theoretical Physics, Yale University, for permission to include the {\tt iopart-num} \BibTeX package (version 2.0, December 21, 2006) with  this documentation. Updates and new releases of {\tt iopart-num} can be found on \verb"www.ctan.org" (CTAN). 
\end{thebibliography}

\end{document}
